%% python-cs.tex
%
% Copyright 2018  Rudy Matela
%
% This text is available under (at your option):
%   * Creative Commons Attribution-ShareAlike 3.0 Licence
%   * GNU Free Documentation License version 1.3 or Later
%


\documentclass{refcard}
\usepackage[T1]{fontenc} % necessary so '{', '}' and '\' get the right font

\renewcommand{\familydefault}{\sfdefault}

\title{Concise Python Cheat Sheet}

\cright{
	Copyright 2018, Rudy Matela --
	Compiled on \today{} \\
	Upstream: \texttt{https://github.com/rudymatela/concise-cheat-sheets}
}{
	This text is available under
	the Creative Commons Attribution-ShareAlike 3.0 Licence, \\
	\textbf{or} (at your option), the GNU Free Documentation License version 1.3 or Later.
}
\version{0.0.0}


\begin{document}

\maketitle

\vspace{4em}
\begin{center}
{\Large\textbf{NOTE:} this is a work in progress}
\end{center}
\vspace{4em}

\section{Structure}

\begin{verbatim}
#!/usr/bin/env python

def hello(who):
    print("Hello " + who + "!");

hello("world")
\end{verbatim}



\section{Literals}

\begin{ldesc}
	\li[String with escapes]    "Escape \textbackslash{}'double-quotes\textbackslash{}', not 'single'"
	\li[String with escapes]    'Escape \textbackslash{}'single-quotes\textbackslash{}', not "double"'
\end{ldesc}


\section{Operators (grouped by precedence)}

\begin{Ldesc}
	\Li[exponentiation]                  **
	\Li[positive, negative, bitwise not] +, -, ~
	\Li[multiplication and division]     *, /, //, \%, @
	\Li[addition and subtraction]        +, -
	\Li[...] ... TODO: add more ...
\end{Ldesc}


\section{Control Flow}

\begin{ldesc}
	\li[if statement]            if \I{condition}: \li
	                             ~~~~\I{code-block} \li

	\li[if-else statement]       if \I{condition1}: \li
	                             ~~~~\I{code-block} \li
								 else: \li
								 ~~~~\I{code-block} \li

	\li[if-elif-else statement]  if \I{condition1}: \li
	                             ~~~~\I{code-block} \li
								 elif \I{condition2}: \li
								 ~~~~\I{code-block} \li
								 else: \li
								 ~~~~\I{code-block} \li

	\li[while statement]         while \I{condition}: \li
	                             ~~~~\I{code-block} \li

	\li[for statement]           for \I{v} in \I{values}: \li
	                             ~~~~\I{code-block} \li

	\li[for statement (copy)]    for \I{v} in values[:]: \li
	                             ~~~~\I{code-block} \li

	\li[for statement (count)]   for \I{i} in range(\I{n}): \li
	                             ~~~~\I{code-block} ~~ \# from 0 to n \li

	\li[break statement]
		\I{while-or-for-statement}: \li
		~~~~break ~~ \# exit loop, skip else \li
		else: \li
		~~~~\I{code-block} \li

	\li[continue-statement]
		\I{while-or-for-statement}: \li
		~~~~continue ~~ \# next loop iteraction \li

	\li[pass-statement (no-op)]
		\I{def do-nothing():} \li
		~~~~pass \li

	\li[case/switch statements]  \textnormal{There aren't any.}
\end{ldesc}


\section{Functions}

\begin{ldesc}
	\li[referencing] \I{fun} \li

	\li[calling]     \I{fun}() \li

	\li[calling (w/arg)] \I{fun}(\I{arg\_value}) \li

	\li[calling (w/kw arg)] \I{fun}(\I{1st\_arg}, \I{2nd\_arg}, \I{key}=\I{val}) \li

	\li[0-arguments] def \I{fun}(): \li
	                 ~~~~\I{code-block} \li

	\li[1-argument]  def \I{fun}(\I{arg}): \li
	                 ~~~~\I{code-block} \li

	\li[2-arguments] def \I{fun}(\I{arg0},\I{arg1}): \li
	                 ~~~~\I{code-block} \li

	\li[default arguments] def \I{fun}(\I{mandatory},\I{optional}=\I{default\_val}): \li
	                       ~~~~\I{code-block} \li

	\li[arbitrary arguments] def \I{fun}(\I{mandatory}, \I{*args}): \li
	                         ~~~~\I{code-block} \li

	\li[lambda (1-arg)]      lambda \I{arg}:~\I{expression} \li

	\li[lambda (2-args)]     lambda \I{arg0},\I{arg1}:~\I{expression} \li

	\li[documentation string] def \I{fun}(\I{args}): \li
	                          ~~~~"""Concise summary of purpose \li
	                          ~~~~ \li
	                          ~~~~Longer description, if needed \li
	                          ~~~~""" \li
	                          ~~~~\I{code-block} \li
\end{ldesc}

\section{Evaluation Order}

Expressions are evaluated left-to-right:
\begin{verbatim}
1st + (2nd * 3rd)
(1st + 2nd) * 3rd
3rd + 4th = 1st, 2nd
\end{verbatim}

\end{document}
